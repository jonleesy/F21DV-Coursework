\documentclass{scrreprt}
\usepackage{scrhack}
\usepackage{listings}
\usepackage{underscore}
\usepackage[margin=2cm]{geometry}
\usepackage{graphicx}
\usepackage[bookmarks=true]{hyperref}
\usepackage[utf8]{inputenc}
\usepackage[english]{babel}
\usepackage[super]{nth}
\usepackage{placeins}
\usepackage[table,xcdraw]{xcolor}
\usepackage{array}
\usepackage{float}
\usepackage{xcolor,listings}
\usepackage{textcomp}
\usepackage{color}

\usepackage{etoolbox}
\makeatletter
\patchcmd{\scr@startchapter}{\if@openright\cleardoublepage\else\clearpage\fi}{}{}{}
\makeatother


\setlength{\parindent}{0em}
\setlength{\parskip}{1.0em}

\definecolor{mygreen}{rgb}{0,0.6,0}
\definecolor{mygray}{rgb}{0.5,0.5,0.5}
\definecolor{mymauve}{rgb}{0.58,0,0.82}
\definecolor{darkgray}{rgb}{.4,.4,.4}
\definecolor{purple}{rgb}{0.65, 0.12, 0.82}

\lstset{ %
backgroundcolor=\color{white}, % choose the background color; you must add \usepackage{color} or \usepackage{xcolor}
basicstyle=\footnotesize, % the size of the fonts that are used for the code
breakatwhitespace=false, % sets if automatic breaks should only happen at whitespace
breaklines=true, % sets automatic line breaking
captionpos=b, % sets the caption-position to bottom
commentstyle=\color{mygreen}, % comment style
deletekeywords={...}, % if you want to delete keywords from the given language
escapeinside={\%*}{*)}, % if you want to add LaTeX within your code
extendedchars=true, % lets you use non-ASCII characters; for 8-bits encodings only, does not work with UTF-8
frame=single, % adds a frame around the code
keepspaces=true, % keeps spaces in text, useful for keeping indentation of code (possibly needs columns=flexible)
keywordstyle=\color{blue}, % keyword style
language=Octave, % the language of the code
morekeywords={*,...}, % if you want to add more keywords to the set
numbers=left, % where to put the line-numbers; possible values are (none, left, right)
numbersep=5pt, % how far the line-numbers are from the code
numberstyle=\tiny\color{mygray}, % the style that is used for the line-numbers
rulecolor=\color{black}, % if not set, the frame-color may be changed on line-breaks within not-black text (e.g. comments (green here))
showspaces=false, % show spaces everywhere adding particular underscores; it overrides 'showstringspaces'
showstringspaces=false, % underline spaces within strings only
showtabs=false, % show tabs within strings adding particular underscores
stepnumber=1, % the step between two line-numbers. If it's 1, each line will be numbered
stringstyle=\color{mymauve}, % string literal style
tabsize=2, % sets default tabsize to 2 spaces
title=\lstname % show the filename of files included with \lstinputlisting; also try caption instead of title
}

\lstdefinelanguage{JavaScript}{
keywords={typeof, new, true, false, catch, function, return, null, catch, switch, var, if, in, while, do, else, case, break},
keywordstyle=\color{blue}\bfseries,
ndkeywords={class, export, boolean, throw, implements, import, this},
ndkeywordstyle=\color{darkgray}\bfseries,
identifierstyle=\color{black},
sensitive=false,
comment=[l]{//},
morecomment=[s]{/*}{*/},
commentstyle=\color{purple}\ttfamily,
stringstyle=\color{red}\ttfamily,
morestring=[b]',
morestring=[b]"
}
\lstset{
language=JavaScript,
extendedchars=true,
basicstyle=\footnotesize\ttfamily,
showstringspaces=false,
showspaces=false,
numbers=left,
numberstyle=\footnotesize,
numbersep=9pt,
tabsize=2,
breaklines=true,
showtabs=false,
captionpos=b
}

\addtokomafont{disposition}{\rmfamily}
%}
\usepackage{hyperref}

\pagenumbering{gobble}

\title{F21DV Coursework Lab 2 Report}
\author{Jonathan Song Yang, Lee (H00255553)}
\date{Demonstrated to: Amit Parekh (11/02/2022)}

\begin{document}

\maketitle

\newpage
\tableofcontents

\pagenumbering{arabic}


\newpage
\chapter{Introduction}
Lab 2 focuses on using \verb|d3.js| for dynamic and interactive visualisation concepts. It was meant to be a step-up from Lab 1, where we were taught the basics of d3.js. 
\par Github Repo: \href{https://github.com/jonleesy/F21DV-Coursework}{https://github.com/jonleesy/F21DV-Coursework}
\par Github Pages: \href{https://jonleesy.github.io/F21DV-Coursework/public}{https://jonleesy.github.io/F21DV-Coursework/public}

\section{Set-up}
The set-up for this lab is the same as the Lab 1, but instead of using hard coded div properties, I have implemented CSS grid for aligning divs and webpage objects, as recommended by my lab 1's lab helper. 
\begin{lstlisting}[language=JavaScript,
    caption={Old Method},
    captionpos=b,
    label={lst:Olddiv}]
    /**
    * Create div's for each question systematically.
    * @param {*} exerciseNumber Task number.
    */
   export function createDiv(exerciseNumber) {
       d3.select('body')
           .append('div')
               .attr('class', 'container')
               .append('div')
                   .attr('class', 'answerCenter')
                   .append('p')
                       .append('strong')
                           .text('Exercise ' + exerciseNumber + ':')
   }
\end{lstlisting}
\begin{lstlisting}[language=JavaScript,
    caption={New Method},
    captionpos=b,
    label={lst:newDiv}]
    /**
    * Similar to createDiv(). Was told to look into 
    * grid instead of using hard coded div settings.
    * This one focuses on that, and will be used starting 
    * from lab2.
    * @param {*} exerciseNumber 
    */
   export function createAnswerDiv(exerciseNumber) {
       d3.select('body')
           .append('div')
               .attr('class', 'grid-container')
               .append('div')
                   .attr('class', 'title-grid')
                   .append('p')
                           .append('strong')
                               .text('Exercise ' + exerciseNumber + ':')
       d3.select('.grid-container')
           .append('div')
           .attr('class', 'answer-grid')
   }
\end{lstlisting}
\begin{lstlisting}[language=JavaScript,
    caption={New CSS Method},
    captionpos=b,
    label={lst:newCSS}]
    /* Part 2 onwards CSS using grid */
    .grid-container {
        display: grid;
        grid-template-columns: auto 100px 100px 100px 100px auto;
        grid-template-rows: 60px auto auto auto auto auto;
        grid-gap: 10px;
        /* background-color: #262c3046; */
        padding: 10px;
    }
    
    .grid-container > div {
        background-color: rgb(226, 238, 240);
        padding: 20px 0;
    }
    
    .title-grid {
        grid-area: 1 / 2 / 1 / 6;
        text-align: left;
        text-indent: 20%;
    }
    
    .answer-grid {
        grid-area: 2 / 2 / 6 / 6;
        text-align: center;
        align-content: center;
    }
    
    .answer-grid-small {
        grid-area: 2 / 3 / 6 / 5;
        text-align: center;
        align-content: center;
    }
\end{lstlisting}
Looking at listing \ref{lst:Olddiv} and \ref{lst:newDiv}, the only difference is with the new grid (\verb|grid-container|) being added, then adding a smaller answer fiv with class \verb|answer-grid| afterwards. The preperties of these grids above are shown in listing \ref{lst:newCSS}.

\newpage
\chapter{Exercises}
\section{Exercise 1}
\begin{figure}[!ht]
    \centering
    \includegraphics[width = 7.5cm]{images/ex1_1.png}
    \includegraphics[width = 7.5cm]{images/ex1_2.png}
    \label{fig:ex1}
    \caption{Exercise 1}
\end{figure}
\FloatBarrier
Figure \ref{fig:ex1} shows a line chart with data points plotted on it. The right figure shows what happens when the mouse hovers upon a data point, the data point would pulse between red and green. *\textit{Can't seem to take a screenshot and capture the pointer}.

\newpage
\section{Exercise 2}
\begin{figure}[!ht]
    \centering
    \includegraphics[width = 7.5cm]{images/ex2_1.png}
    \includegraphics[width = 7.5cm]{images/ex2_2.png}
    \label{fig:ex2}
    \caption{Exercise 2}
\end{figure}
\FloatBarrier
% \lstinputlisting[language=JavaScript]{../../public/js/part2/task2.js}
Figure \ref{fig:ex2} shows 4 blocks with different colours. The blocks are just a div defined within a grid. Each block also has a fill colour defined using the \verb|Rgb()| colour method, as shown in line 10. Upon hovering the mouse on the div, the div would show a text element displaying the Rgb colour. This effect is done using the css \verb|:hover| method.

\newpage
\section{Exercise 3}
\begin{figure}[!ht]
    \centering
    \includegraphics[width = 7.5cm]{images/ex3_1.png}
    \includegraphics[width = 7.5cm]{images/ex3_2.png}
    \label{fig:ex3}
    \caption{Exercise 3}
\end{figure}
\FloatBarrier
\lstinputlisting[language=JavaScript]{../../public/js/part2/task3.js}
Figure \ref{fig:ex3} shows a block thats green colour. Upon mouse hover on the box, the box now shows a text that says `Pointer is in the box', its colour is now orange, and it has a new type of border. This is done using the \verb|.on()| function.

\newpage
\section{Exercise 4}
\begin{figure}[!ht]
    \centering
    \includegraphics[width = 7.5cm]{images/ex4_1.png}
    \includegraphics[width = 7.5cm]{images/ex4_2.png}
    \label{fig:ex4}
    \caption{Exercise 4}
\end{figure}
\FloatBarrier
% \lstinputlisting[language=JavaScript]{../../public/js/part2/task4.js}
Figure \ref{fig:ex4} shows an SVG object with a circle in the middle. Upon hovering on the circle, the circle enlarges. This time, the action was modeled using d3 transitions instead of css :hover method. I've added transtitions upon mouse hover and out.

\newpage
\section{Exercise 4}
\begin{figure}[!ht]
    \centering
    \includegraphics[width = 7.5cm]{images/ex5.png}
    % \includegraphics[width = 7.5cm]{images/ex4_2.png}
    \label{fig:ex5}
    \caption{Exercise 5}
\end{figure}
\FloatBarrier
% \lstinputlisting[language=JavaScript]{../../public/js/part2/task5.js}
Figure \ref{fig:ex4} shows an emptey svg object, and upon mouse hover, it would show the corredinated of the mouse. There was also a pre-appended empty text box. To show the x-y coordinates, this is done using the event data of the mouse movement. Then using the data, I modified the `x' and `y' attribute of the text box.

\end{document}